\documentclass{article}

\usepackage{tikz}
\usetikzlibrary{arrows,backgrounds,patterns,matrix,shapes,fit,calc,shadows,plotmarks,decorations,decorations.text,shapes.gates.logic.US}
\usepackage{pgf}

\usepackage{verbatim}
\usepackage[tightpage,active]{preview}
\PreviewEnvironment{tikzpicture}
\setlength\PreviewBorder{.5mm}

\newcommand{\chan}[1]{\ensuremath{\mathcal{#1}}}

% Load Math commands
\newcommand{\IC}{{\mathcal I}}
\newcommand{\CC}{{\mathcal C}}
\newcommand{\lm}{\lambda}

\begin{document}
\pagestyle{empty}

\begin{comment}
:Title: Incomplete Erasure Channel
\end{comment}

\tikzstyle{conBox} = [draw=black, fill=white,rounded rectangle, minimum height=.7cm, minimum width=1cm]
\tikzstyle{Textbox} = [draw=black, fill=black!60!green!60, rounded rectangle, minimum height=.8cm, minimum width=.8cm]
\tikzstyle{openC} = [draw=black, fill=white, circle, inner sep = 0pt, outer sep=0pt, minimum size=7pt]
\tikzstyle{closeC} = [draw=black, fill=black, circle, inner sep = 0pt, outer sep=0pt, minimum size=7pt]

%Draw Pipe
\tikzstyle{chanPipe}=[cylinder, draw=black,thick, aspect=.25, minimum height=5cm,
                      minimum width=2.9cm, cylinder uses custom fill, 
                      cylinder body fill=blue!15, cylinder end fill=blue!5 ]

\tikzstyle{chanPipeI}=[cylinder, draw=black,thick, aspect=.25, minimum height=3cm,
                      minimum width=1cm, cylinder uses custom fill, 
                      cylinder body fill=white, cylinder end fill=white]

\tikzstyle{chanPipeC1}=[cylinder, draw=black,thick, aspect=.25, minimum height=3cm,
                      minimum width=1cm, cylinder uses custom fill, 
                      cylinder body fill=blue!50, cylinder end fill=blue!20]

%Draw Arc in Pipe
\newcommand\drawArc[1]{
\draw[dashed]
let \p1 = ($ (#1.after bottom) - (#1.before bottom) $),
\n1 = {0.5*veclen(\x1,\y1)-\pgflinewidth},
\p2 = ($ (#1.bottom) - (#1.after bottom)!.5!(#1.before bottom) $),
\n2 = {veclen(\x2,\y2)-\pgflinewidth}
in
([xshift=-\pgflinewidth] #1.before bottom) arc [start angle=270, delta angle=180,
x radius=\n2, y radius=\n1]
}

\newcommand\drawArcL[1]{
\draw[dashed]
let \p1 = ($ (#1.after bottom) - (#1.before bottom) $),
\n1 = {0.5*veclen(\x1,\y1)-\pgflinewidth},
\p2 = ($ (#1.bottom) - (#1.after bottom)!.5!(#1.before bottom) $),
\n2 = {veclen(\x2,\y2)-\pgflinewidth}
in
([xshift=-\pgflinewidth] #1.before bottom) arc [start angle=270, delta angle=180,
x radius= .1cm, y radius=\n1]
}


\begin{tikzpicture}[auto,>=latex']

    %Input
    \node [conBox] at (-1.25,1) (A) {$A$};
    
    %Channel
    %Big Pipe
    \node [chanPipe] at (2,1) (CC) {};
    \drawArcL{CC};

    %\node [] at (2,-1) (cap) {Incomplete erasure channel $\CC$};

    %Small Pipe Up
    \node [chanPipeC1] at (2.5,1.65) (CC1) {$\CC_1$};
    \drawArc{CC1};
    
    %Small Pipe Down
    \node [chanPipeI] at (2.5,.35) (IC) {$\IC$};
    \drawArc{IC};
 
    %Input to Big Pipe
    \draw[->] (-.86,1) -- (-.5,1);
    %Big Pipe to Pipe Up and Down
    \draw[->, dash dot] (-.5,1) -- (1,1.65);
    \node[] at (0.25,1.625) (Mlm) {$1-\lm$};
    \draw[->, dash dot] (-.5,1) -- (1,.35);
    \node[] at (0.25,0.4) (lm) {$\lm$};

    %Pipe Up and Down to Output
    \draw[->] (4,1.65) -- (5,1.65);
    \draw[->] (4,.35) -- (5,.35);
    
    %Output
    \draw [draw=blue!70, rounded corners=5pt] (5,-.25) rectangle (6,2.25);
    \draw [draw=blue!70] (5,1) -- (6,1);

    \node [] at (5.5,1.65) (CC1A) {$\CC_1(A)$};
    \node [] at (5.5,.35) (ICA) {$A$};
\end{tikzpicture}

\end{document}

